\documentclass[12pt, a4paper]{article}

\title{PE Autumn 2017 -- Statechart}
\begin{document}
\maketitle

\section{Introduction}
This is the project charter for the project elective for Harika.

\section{Goals and Objective}
To implement modules on:
\begin{enumerate}
	\item Front end for Statechart
	\item Define data structures for representation of abstract syntaxt tree (AST) and control flow graph (CFG)
	\item Implement algorithm to derive CFG from AST
	\item Implement data flow analysis algorithm to demonstrate the ability of the algorithm to detect specification bugs involving undefined variable errors on identified case studies
	\item Prepare a ready-to-download package (code, documentation and case studies) to be uploaded in GitHub
\end{enumerate}

\section{Description}
The project implementation work has to be carried out in close partnership with Karthika Venkatesan\footnote{\texttt{karthika.venkatesan@iiitb.ac.in}, \texttt{+91 98443 34898}}. Her responsibility will be to
\begin{enumerate}
	\item focus on the detailed design of all the algorithmic and architectural aspects of the tool
	\item take lead to make ready the case studies on which the tool will be tested
	\item track the progress closely and report any issue or concern on a timely basis.
\end{enumerate}

The project involves the following stages:
\subsection{Introduction}
In this, the student will develop understanding of the problem, and take handover of any knowledge and resources already existing.

\subsection{Front end}
In this, the front end for Statechart has to be implemented. This involves a parser that can read a statechart written in a language form and generate a control flow graph for the same to be given as as the input to the data flow analysis algorithm.

\subsection{Data structures}
In this, two modules have to be implemented as follows:


\subsubsection{Abstract syntax tree}
In this part, the object oriented language library to represent the abstract syntax tree for Statechart representation has to be implemented.

\subsubsection{Control flow graph}
In this part, object oriented library to represent a control flow graph will be developed.

\subsection{Algorithm to convert AST to CFG}
In this part, the algorithm to convert the abstract syntax tree to the equivalent control flow graph will be implemented.

\subsection{Data flow analysis}
In this part, the data flow analysis algorithm (reaching definition) will be implemented to identify undefined variable bugs in the specifications.

\subsection{Experiment}
The proof of the concept will be demonstrated using homemade statechart specifications. The tool will be run on these specifications, and bugs -- either naturally occuring or injected by design -- will be identified automatically.

\subsection{Downloadable package}
The entire work done should culminate in a software package that can be downloaded and the experiments can be repeated at the push of a button by anyone.

\section{Tentative Timeline}

\begin{center}
\begin{tabular}{| l | l | l |}
\hline
\multicolumn{2}{| c |}{\textbf{Timespan}} & \textbf{Deliverable} \\
\hline
\multicolumn{1}{| c |}{\textbf{Start}} & \multicolumn{1}{| c |}{\textbf{End}} &  \\
\hline
Aug. 10, 2017 & Aug. 15, 2017 & Introduction \\
\hline
Aug. 15, 2017 & Aug. 31, 2017 & Front end + AST \\
\hline
Sep. 01, 2017 & Sep. 15, 2017 & AST to CFG \\
\hline
Sep. 15, 2017 & Sep. 30, 2017 & Data flow Analysis \\
\hline
Oct. 01, 2017 & Oct. 15, 2017 & Experiment \\
\hline
Oct. 15, 2017 & Oct. 31, 2017 & Package \\
\hline
Nov. 01, 2017 & Nov. 15, 2017 & Improvements \\
\hline
Nov. 15, 2017 & Nov. 30, 2017 & Buffer \\
\hline
\end{tabular}
\end{center}

\section{Modus Operandi}
\begin{enumerate}
	\item There will at least one update meeting every week on a mutually convenient time slot, typically lasting for about half an hour.
	\item These meetings will be used to give updates, plan for the subsequent week and discuss issues if any.
	\item There will be more meetings if required.
	\item Research student will keep a day-to-day tap on the progress and meet at least one more time a week on average with the project student.
	\item Meeting cancellations, if any, must happen with prior notice.
	\item Regular attendance of meeting is an important component of the project assessment.
	\item All progress, thoughts, ideas discussed will be documented on a regular basis.
\end{enumerate}
\end{document}